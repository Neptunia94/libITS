\section{Introduction}\label{sec:introduction}

Instantiable Transition Systems are intended as a low level
description of complex systems semantics. Their semantics is defined
in the simplest manner possible, using labeled states and
transitions. However, through hierarchical composition and the use
of partial synchronization functions to compose behaviors, the
framework is rich enough to capture the complex semantics of modern
behavioral modeling formalisms (e.g. UML,AADL\ldots).

Preserving the structure of the specification down to the semantic
level, instead of defining the semantics through a flattening,
allows to use this information in the solution algorithms and
data-structures.

In particular, the definition of ITS is partly driven by the concern
of preserving optimal conditions to use SDD as a powerful support
for symbolic model-checking.

Several ITS types are available in the library, and the design
allows to easily add new types.

We first present the formal definition of ITS and we briefly
introduce SDD and their homomorphisms, that are manipulated to
implement the semantics of a given formalism.

From the ITS modeler perspective, the sections on SDD can be safely
skipped, they are relevant as a reference and a comparison when
implementing new ITS types.

We then present the various ITS types or formalisms currently
supported by the library. For each formalism we give the syntax, the
semantics through an ITS definition, at least one example, and the
SDD encoding.

We first introduce the Composite type offered as a general
structuring mechanism. It is a basis to defined a generalized
synchronized product of subsystems.

Then we present Labeled Transition Systems, as a very simple example
of an elementary type.

We then present the variations on the Composite type used to
represent regular models: the ScalarSet and CircularSet. Regular
models designates models exhibiting a form of structural behavioral
symmetry. This allows a more compact and parametric model
definition. The solution engine is also able to exploit these
symmetries.

The variant of Petri nets supported by the library is then
presented. Based on the nets used in the tool Romeo, they support
many powerful structural features such as inhibitor arcs, test arcs,
reset arcs and hyper-arcs. They also support discrete time, through
time bounds on transition "� la" Tina.

Still in the timed model category, we present the Timed Composite as
a variation on the Composite definition to model composite timed
systems. Using this definition, one can compose timed Petri nets
with other timed formalisms that use discrete semantics.

For instance with Timed Automata, a popular formalism used in Uppaal
or Kronos. The TA that are supported in the library offer discrete
variables with arithmetic as well as clocks and locations. Time
constraints can use non-strict inequalities with integer constants.

To allow fully symbolic LTL model-checking, the encoding of a
Transition-based Generalized B�chi Automata (TGBA) is presented
next. This formalism is the central one used in the model-checking
library Spot (A. Duret-Lutz and D. Poitrenaud). By definition of an
appropriate synchronization through a composite, this enables LTL
model-checking.

ITS offer support for LTL (using components from Spot) and CTL
(using components from VIS). The usage of these tools is described
next.

Finally an annex is provided describing the various ways to input
ITS. In particular the Coloane ITS plugin, and the various file
formats supported are described. A user-oriented API can also be
used to create ITS definitions programmatically, but this is
described in the technical doxygen generated pages.
