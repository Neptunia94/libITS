\section{Technical considerations}

\textbf{Building IPN:}
IPN are meant to be used as a powerful back-end verification engine
rather than a front-end modeling language. The semantics are simple
and complex data types (arrays, lists...) are not directly supported.
For simple examples, the file format presented in
section~\ref{ss:philo} has been defined, but mostly we expect users to
create IPN from higher level modeling languages through a translation
approach. To this end, an easily manipulable API has been defined and
implemented both in Java and C++.

The main alternative would have been to offer programming language
constructs (loops, if/then/else, arrays, string operations to build
labels based on patterns etc\ldots) directly within the IPN file
format. While viable in theory, we think this approach is difficult to follow
because $a)$ users have to learn a (new) complex language $b)$ we
would have to implement a complex parser, which to be usable needs to
offer good support for syntax errors and is liable to introduce bugs
and needless complexity in the supporting tools. So, we chose instead 
to offer a simple API, and let the user profit from all the facilities
 of his favorite language, among which the convenience of object-oriented
 features would have been out of scope for a dedicated language anyway.

The Java language is supported because of its widespread use, and the
support for model-driven use case scenarios that is provided by the
Eclipse Modeling Framework (EMF) \cite{}. 
%EMF allows to easily define
% and manipulate metamodels, which is a prerequisite when defining 
%model-to-model transformations using modern technologies. In particular,
In particular, it offers good support for the manipulation of OMG standard compliant 
 UML2 models, which was a design goal when constructing this tool set.

The actual verification engine relies on \texttt{libddd} which is a
C++ library built for memory efficiency and execution speed. It is
thus natural for us to also offer a similar IPN construction API in
C++, as it is the implementation language of choice for the
computation heavy procedures of model-checking. 

Because direct sharing of code and memory between C++ and Java is
difficult, interaction between the two languages is handled through
files written (or parsed) in the IPN file format. We use a common
ANTLR \cite{} grammar to parse these files in both languages.  This
file format is meant to be readable by humans; an alternative would
have been to use XML/XMI file formats as generated from the IPN metamodel
by EMF, but their large size and the fact that human users cannot
read them without great difficulty made us discard this option.


\fixme{Parler du metamodele EMF des IPN defini par Lom ? De la compatibilit� par aplatissement de la hierarchie sur le format CAMI ? A travers ce biais de la possibilit� d'exporter sur PNML ?}


\textbf{Model transformation technology:} We have opted for
direct manipulation of the UML2 API and of the IPN API (in Java) to construct
the target models.  The alternatives we considered were $a)$ using an
implementation of QVT standard \cite{} but we did not find one which
matched our needs, so we discarded this option quite rapidly $b)$
using the ATL \cite{} transformation language and engine ATLAS.
This option we actually prototyped, but
we had non-trivial issues with being able to trace the transformation
rules applied, which we need to interpret the verification results in
terms of the original UML specification. Furthermore, in ATL the user
has no control over application of transformation rules, and the
technology is not yet quite up to industrial strength. Lastly it requires
learning a new and quite complex language.

The option we selected of directly using Java to implement the
transformation allows us to fully master both the transformation
process and the construction of trace information, and our existing
expertise with Java made this direction the easiest to follow in our
setting.
